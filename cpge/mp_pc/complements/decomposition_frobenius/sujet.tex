\documentclass[a4paper,11pt]{article}

%-----------------------------------------------------------------

%Packages utilisés
\usepackage{amsmath, amssymb, amsthm}	%Packages mathématiques
\usepackage[utf8]{inputenc}				%Encodage UTF8
\usepackage[francais]{babel}				%Typographie française
\usepackage{fourier}						%Police Fourier
\usepackage{fancyhdr}					%En-tête personnalisé
\usepackage[margin=.8in]{geometry}		%Modification des marges
\usepackage[usenames, dvipsnames]{color, xcolor}	%Ajout de couleurs
\usepackage{titlesec}					%Formattage des titres

%Personnalisation de l'en-tête/pied de page
\pagestyle{fancy}
\renewcommand{\headrulewidth}{0pt}
	%En-tête
	\fancyhead[L]{%
		\strut\rlap{
			\color{vert1}
			\rule[-\dp\strutbox]{\headwidth} {\headheight}
		}%
	  	\sffamily{\color{gris}{ Lycée J-B Corot}}
	}
	\fancyhead[C]{\sffamily{\color{gris}{MP}}}
	
	%Insérer le titre ICI
	\fancyhead[R]{\sffamily{\color{gris}{Autour des matrices de Frobenius}}}
	%--------------------------------------------------
	
	%Pied de page
	\fancyfoot[L]{%
		\strut\rlap{
			\color{vert1}
			\rule[-\dp\strutbox]{\headwidth}{\headheight}
		}%
	  	\sffamily{\color{gris}{}}
	}
	\fancyfoot[C]{%
		\strut\rlap{
			\rule[-\dp\strutbox]{\headwidth}{0pt}
		}%
	  	\sffamily{\color{gris}{\thepage}}
	}
	\fancyfoot[R]{\sffamily{\color{gris}{}}}
	
%Nouvelles commandes
	%Couleurs
	\definecolor{gris}{HTML}{EEEEEE}
	\definecolor{vert1}{HTML}{6AB7C3}
	
	%Numérotation
	\renewcommand{\labelenumi}{\sffamily{\textbf{\color{vert1}\theenumi.}}}
	\renewcommand{\labelenumii}{\sffamily{\textbf{\color{vert1}\theenumii.}}}
	\renewcommand{\labelenumiii}{\sffamily{\textbf{\color{vert1}\theenumiii.}}}
	
	%Mathématiques
	\newcommand{\N}{\mathbf{N}}
	\newcommand{\Z}{\mathbf{Z}}
	\newcommand{\R}{\mathbf{R}}
	\newcommand{\C}{\mathbf{C}}
	\newcommand{\Q}{\mathbf{Q}}
	\newcommand{\K}{\mathbf{K}}
	\newcommand{\U}{\mathbf{U}}
	\renewcommand{\P}{\mathbf{P}}
	\newcommand{\D}{\mathcal{D}}
	\newcommand{\E}{\mathbf{E}}
	\newcommand{\M}{\mathcal{M}}
	\newcommand{\B}{\mathcal{B}}
	\newcommand{\CC}{\mathcal{C}}
	\renewcommand{\L}{\mathcal{L}}
	\newcommand{\suite}[2]{{\left({#1}_{#2}\right)}_{#2}}
	
	%Titre
	\newcommand{\titre}[1]{
		\colorbox{vert1}{
		\parbox{\headwidth}{
			\center{\Large{\sffamily\bfseries{\color{gris}{#1}}}}\\
			\vspace*{1em}
		}}
	}
	
	%Corrigé
	\newcommand{\corrige}[1]{
		\colorbox{vert1}{
		\parbox{\headwidth}{
			{\sffamily\bfseries{\color{gris}{Théorème}}}\par
			{#1}\\
		}
		}
		
	}
	
	%Fin
	\newcommand{\fin}{
	\begin{center}
	\vspace*{2em}
	{\color{vert1}\rule{5cm}{0.2pt}}\\
	\vspace*{1em}
			\sffamily\bfseries{\color{vert1}{FIN DU SUJET}}
	\end{center}
	}
	
	%Formattage des titres
	\renewcommand{\thesection}{\Roman{section}}
	\renewcommand{\thesubsection}{\Alph{subsection}}
	\titleformat{\section}
  		{\normalfont\sffamily\Large\bfseries\color{vert1}}
  		{\thesection .}{1em}{}
  	\titleformat{\subsection}
  		{\normalfont\sffamily\large\bfseries\color{vert1}}
  		{\hspace{2em}\thesubsection .}{.5em}{}

%Autres
\setlength\parindent{0pt}	%Pas d'indentation des paragraphes 

%-----------------------------------------------------------------

\begin{document}
\thispagestyle{empty}				%Pas d'en-tête sur la page 1
\pagecolor{gris}						%Couleur de fond
\titre{Autour des matrices de Frobenius}		%Titre du document
\vspace*{1em}

\section*{Notations}
Dans tout le problème, $\K$ désignera le corps $\R$ ou $\C$. On considère un $\K$-espace vectoriel $E$ de dimension finie $n\in\N^*$, muni d'un produit scalaire $(\cdot\vert\cdot)$. On désigne par $\M_n(\K)$ l'ensemble des matrices carrées d'ordre $n$ à coefficients dans $\K$, et par $\mathrm{GL}_n(\K)$ l'ensemble des matrices inversibles de $\M_n(\K)$. On pose $I_n$ la matrice identité de $\M_n(\K)$. On note enfin $\K[X]$ l'anneau des polynômes à coefficients dans $\K$.\\

À tout polynôme $P=\displaystyle\sum_{k=0}^na_kX^k\in\K[X]$ unitaire (i.e. $a_n=1$), on associe sa matrice compagnon $C_P$, définie par
\[C_P=
\begin{pmatrix}
0&0&\cdots&0&-a_0\\
1&0&\cdots&0&-a_1\\
0&1&\cdots&0&-a_2\\
\vdots&\ddots&\ddots&\vdots&\vdots\\
0&\cdots&0&1&-a_{n-1}
\end{pmatrix}.\]

En notant $a=(a_0,\cdots,a_{n-1})$, on pose $C_a=C_P$. Dans tout le problème, on s'autorisera à confondre ces deux notations.

Pour tout endomorphisme $\varphi\in\L(E)$ (resp. $M\in\M_n(\K)$), on note respectivement $\chi_\varphi$ et $\pi_\varphi$ (resp. $\chi_M$ et $\pi_M$) ses polynômes caractéristique et minimal. Pour $M\in\M_n(\C)$, on note $M^\top$ sa transposée. Si $F$ est un sous-espace vectoriel de $E$ stable par $\varphi$, on note $\varphi_{\vert F}$ la restriction de $\varphi$ à $F$. Enfin, si $\B$ est une base de $E$ et $\varphi\in\L(E)$, on note $M_\B(\varphi)$ la matrice de $\varphi$ dans la base $\B$.\\

Pour $\varphi\in\L(E)$, on note $\K[\varphi]=\{P(\varphi)\ \vert\ P\in\K[X]\}$.\\

Un endomorphisme $\varphi$ est dit cyclique s'il existe $x\in E$ tel que $\B=\left\lbrace x,\varphi(x),\cdots,\varphi^{n-1}(x)\right\rbrace$ soit une base de $E$. Une matrice $M\in\M_n(\K)$ est dite cyclique si elle est la matrice d'un endomorphisme cyclique. On note $\CC_n$ l'ensemble des matrices cycliques.\\

Pour tout sous-espace vectoriel $F$ de $E$, on note $F^\perp$ l'orthogonal de $F$, c'est-à-dire l'ensemble $\{x\in E\ \vert\ \forall y\in F,\ (x\vert y)=0\}$.\\

On pose $E^*=\L(E,\K)$. Pour $1\leq i\leq n$, $\B=\{e_1,\cdots, e_n\}$ une base de $E$, on note $e^*_i$ l'élément de $E^*$ (on ne demande pas de le vérifier) défini par
\[\forall x=\displaystyle\sum_{k=1}^nx_ie_i\in E,\ \ e^*_i(x)=x_i,\]
et on pose $\B^*=\{e^*_1,\cdots, e^*_n\}$.\\

Pour tout $x\in E$, on note $\phi_x$ l'élément de $E^*$ défini par
\[\forall y\in E,\ \phi_x(y)=(x\vert y),\]
et on pose 
\begin{align*}
\Phi:\ &E\rightarrow E^*\\
&x\mapsto\phi_x
\end{align*}

Enfin, si $A$ est un sous-espace vectoriel de $E^*$, on note $A^\circ$ l'ensemble $\{x\in E\ \vert\ \forall\phi\in A,\ \phi(x)=0\}$.\\

On s'autorisera dans tout le problème la confusion entre matrice et endomorphisme (au sens où, par exemple, on pourra considérer un endomorphisme $\varphi\in\CC_n$).\\

\section{Préliminaires}
\begin{enumerate}
\item Montrer que $\Phi$ est un isomorphisme. En déduire la dimension de $E^*$.
\item Soit $F$ un sous-espace vectoriel de $E$, de dimension $d$.
\begin{enumerate}
\item Montrer que $E=F\oplus F^\perp$.
\item En déduire la dimension de $F^\perp$.
\end{enumerate}
\item Soit $A$ un sous-espace vectoriel de $E^*$, de dimension $d$.
\begin{enumerate}
\item Montrer que $A^\circ=\left(\Phi^{-1}(A)\right)^\perp$.
\item En déduire que la dimension de $A^\circ$ est $n-d$.
\end{enumerate}
\item Soit $\varphi\in\L(E)$, $F$ et $G$ deux sous-espaces vectoriels supplémentaires dans $E$ stables par $\varphi$. Montrer que, pour tout $P\in\K[X]$, $P(\varphi)(E)=P(\varphi)(F)\oplus P(\varphi)(G)$.
\end{enumerate}
\section{Endomorphismes et matrices cycliques}
Dans cette sous-partie, on considère $\varphi\in\L(E)$.\\
\begin{enumerate}
\setcounter{enumi}{4}
\item Montrer que, pour tout polynôme unitaire $P\in\K[X]$, $\chi_{C_P}=P$.
\item 
\begin{enumerate}
\item Soit $\B$ une base de $E$. Montrer que, si $M_\B(\varphi)$ est une matrice compagnon, alors $\varphi$ est cyclique.
\item On suppose que $\varphi$ est cyclique et on considère $x\in\E$ tel que $\B=\left\lbrace x,\varphi(x),\cdots,\varphi^{n-1}(x)\right\rbrace$ soit une base de $E$.
\begin{enumerate}
\item Montrer qu'il existe $P\in\K[X]$ tel que $M_\B(\varphi)=C_P$.
\item En déduire que toute matrice cyclique est semblable à une matrice compagnon.
\end{enumerate}
\end{enumerate}
\item Pour tout $x\in E$, on note $I_{\varphi,x}=\{P\in\K[X]\ \vert\ P(\varphi)(x)=0\}$.
\begin{enumerate}
\item Montrer que, pour tout $x\in E$, $I_{\varphi,x}$ est un idéal de $\K[X]$.
\item En déduire que, pour tout $x\in E$, il existe un unique $\pi_{\varphi,x}\in\K[X]$ unitaire tel que $I_{\varphi,x}=\{\pi_{\varphi,x}\cdot P\ \vert\ P\in\K[X]\}$.
\item On veut montrer qu'il existe $x\in E$ tel que $\pi_\varphi=\pi_{\varphi,x}$.
\begin{enumerate}
\item Montrer que, pour tout $x\in E$, $\pi_\varphi\in I_{\varphi,x}$ et en déduire que $\pi_{\varphi,x}\vert\pi_\varphi$.
\item Montrer qu'il existe $x_1,\cdots, x_k\in E$ tels que $E=\displaystyle\bigcup_{i=1}^k\mathrm{Ker}\left(\pi_{\varphi,x_i}(\varphi)\right)$.
\item Montrer qu'il existe $1\leq i\leq k$ tel que $E=\mathrm{Ker}\left(\pi_{\varphi,x_i}(\varphi)\right)$. En déduire que $\pi_\varphi\vert\pi_{\varphi,x_i}$.
\item Conclure.
\end{enumerate}
\end{enumerate}
\item On se propose de montrer que $\varphi$ est cyclique si, et seulement si, $\chi_\varphi=\pi_\varphi$.
\begin{enumerate}
\item Montrer que, si $\varphi$ est cyclique, alors $\pi_\varphi$ est de degré $n$. En déduire que $\chi_\varphi=\pi_\varphi$.
\item On suppose que $\chi_\varphi=\pi_\varphi$.
\begin{enumerate}
\item Montrer qu'il existe $x\in E$ tel que tout polynôme $P\in I_{\varphi,x}$ non nul soit de degré supérieur ou égal à $n$.
\item Montrer que $\B=\left\lbrace x,\varphi(x),\cdots,\varphi^{n-1}(x)\right\rbrace$ est une base de $E$. Conclure.
\end{enumerate}
\item Soit $P\in\K[X]$. Déterminer $\pi_{C_P}$.
\end{enumerate}
\end{enumerate}

\section{Théorème de décomposition de Frobenius}
L'objectif de cette sous-partie est de démontrer le théorème suivant:\\

Soit $\varphi\in\L(E)$. Il existe des sous-espaces vectoriels $E_1,\cdots,E_r$ de $E$, tous stables par $\varphi$, tels que 
\begin{itemize}
\item[•] $E=\displaystyle\bigoplus_{i=1}^rE_i$
\item[•] Pour tout $1\leq i\leq r$, $\varphi_i=\varphi_{\vert E_i}$ est un endomorphisme cyclique
\item[•] Pour tout $1\leq i\leq r-1$, $\pi_{i+1}=\pi_{\varphi_i}\vert\pi_i$.
\end{itemize}
La suite des $\pi_i$ est alors définie de manière unique. En particulier, il existe une base $\B$ de $E$ telle que
\[M_\B(\varphi)=
\begin{pmatrix}
C_{\chi_1}\\
&\ddots\\
&&C_{\chi_r}
\end{pmatrix}.\]

On considère donc $\varphi\in\L(E)$ et on note $d$ le degré de $\pi_\varphi$.
\begin{enumerate}
\setcounter{enumi}{8}
\item Soit $y\in E$. On note $E_y=\left\lbrace P(\varphi)(y)\ \vert\ P\in\K[X]\right\rbrace$. Montrer que $E_y$ est le plus petit sous-espace vectoriel de $E$ stable par $\varphi$ et contenant $y$.
\item Montrer qu'il existe $y\in E$ tel que $E_y$ soit de dimension $d$, et que $\left\lbrace y,\varphi(y),\cdots,\varphi^{d-1}(y)\right\rbrace$ est une base de $E_y$. On note alors $e_i=\varphi^{i-1}(y)$ pour tout $1\leq i\leq d$.
\item On note $F=\left\lbrace x\in E\ \vert\ \forall k\in\N,\ e^*_d\left(\varphi^k(x)\right)=0\right\rbrace$. 
\begin{enumerate}
\item Montrer que $F$ est stable par $\varphi$.
\item Montrer que $E_y\cap F=\{ 0\}$.
\item On veut montrer que $\mathrm{dim}F=n-d$. Pour cela, on considère l'opérateur
\[T_\varphi:\ \K[\varphi]\rightarrow E^*\ ,\ g\mapsto e^*_d\circ g.\]
\begin{enumerate}
\item Montrer que $T_\varphi$ est injectif. En déduire le rang de $T_\varphi$.
\item Montrer que $\left(\mathrm{Im}T\right)^\circ=F$.
\item Conclure.
\end{enumerate}
\item En déduire que $E=E_y\oplus F$.
\end{enumerate}
\item Posons $\pi_1=\pi_{\varphi_{\vert E_y}}$ et $\pi_2=\pi_{\varphi_{\vert F}}$.
\begin{enumerate}
\item Justifier que l'on peut ainsi définir ces deux polynômes.
\item Montrer que $\pi_1=\pi_\varphi$.
\item En déduire que $\pi_2\vert\pi_1$.
\end{enumerate}
\item Démontrer l'existence de sous-espaces vectoriels $E_1,\cdots,E_r$ de $E$ satisfaisant aux conditions du théorème.
\item Soit $F_1,\cdots,F_r$ et $G_1,\cdots,G_s$ des sous-espaces vectoriels non triviaux de $E$, tous stables par $\varphi$, satisfaisant aux conditions du théorème. On note $\pi_i=\pi_{\varphi_{\vert F_i}}$ et $\psi_j=\pi_{\varphi_{\vert G_j}}$ pour $1\leq i\leq r$ et $1\leq j\leq s$. On veut montrer que $r=s$ et que, pour tout $1\leq i\leq r$, $\pi_i=\psi_i$.
\begin{enumerate}
\item Justifier que $\pi_1=\psi_1$.
\item On raisonne par l'absurde et on note $j\geq 2$ le plus petit entier tel que $\pi_j\neq\psi_j$.
\begin{enumerate}
\item Montrer que 
\[\pi_j(\varphi)(E)=\pi_j(\varphi)(G_1)\oplus\cdots\oplus \pi_j(\varphi)(G_s).\]
\item Montrer que
\[\pi_j(\varphi)(E)=\pi_j(\varphi)(F_1)\oplus\cdots\oplus \pi_j(\varphi)(F_{j-1}).\]
\item Montrer que, pour tout $1\leq i<j$, $\mathrm{dim}\pi_j(\varphi)(F_i)=\mathrm{dim}\pi_j(\varphi)(G_i)$.

{\textit{\footnotesize On pourra commencer par montrer que $\varphi_{\vert F_i}$ et $\varphi_{\vert G_i}$ sont semblables, pour tout $1\leq i<j$.}}
\item En déduire que $\mathrm{dim}\pi_j(\varphi)(G_i)=0$, pour $j\leq i\leq s$, puis que $\psi_j\vert\pi_j$.
\item Aboutir à une contradiction.
\end{enumerate}
\item Conclure.
\end{enumerate}
\end{enumerate}
\section{Quelques propriétés topologiques}
Dans cette partie, $\K$ désignera le corps $\C$.


On munit $E$ de la norme $\Vert\cdot\Vert$ induite par le produit scalaire $(\cdot\vert\cdot)$, et $\L(E)$ (ou, de même, $\M_n(\C)$) d'une norme $N(\cdot)$.
\begin{enumerate}
\setcounter{enumi}{14}
\item Montrer que $\CC_n$ est un ouvert de $\M_n(\C)$.

{\textit{\footnotesize Pour $A\in\M_n(\C)$, on pourra considérer l'application continue $\varphi_x:\ M\in\M_n(\C)\mapsto\det\left(x,Mx,\cdots, M^{n-1}x\right)$, où $x$ sera judicieusement choisi en fonction de $A$.}}
\item 
\begin{enumerate}
\item Montrer que $\mathrm{GL}_n(\C)$ est connexe par arcs.

{\textit{\footnotesize On pourra relier les matrices triangulaires supérieures à $I_n$ dans un premier temps.}}
\item En déduire que $\CC_n$ est connexe par arcs.
\end{enumerate}
\item
\begin{enumerate}
\item Soit $M\in\M_n(\C)$ possédant $n$ valeurs propres distinctes. Montrer que $M$ est semblable à une matrice compagnon.
\item Montrer que l'ensemble des matrices de $\M_n(\C)$ possédant $n$ valeurs propres distinctes est dense dans $\M_n(\C)$.
\item En déduire que $\CC_n$ est dense dans $\M_n(\C)$.
\end{enumerate}
\item Montrer que l'application $\varphi:\ A\in\M_n(\C)\mapsto\pi_A$ n'est pas continue.

{\textit{\footnotesize On pourra raisonner par l'absurde et utiliser, après l'avoir montrée, la continuité de $\psi:\ A\in\M_n(\C)\mapsto\chi_A$.}}
\end{enumerate}

\section{Propriétés spectrales}
Dans cette partie, on fixe $P\in\K[X]$.
\begin{enumerate}
\setcounter{enumi}{18}
\item Montrer que les sous-espaces propres de $C_P$ sont de dimension $1$.
\item On suppose que $P$ est scindé à racines simples, et on note $\lambda$ une racine de $P$.
\begin{enumerate}
\item Montrer que $\lambda$ est une valeur propre de ${C_P}^\top$.
\item Déterminer un vecteur propre $e_\lambda$ de ${C_P}^\top$ associé à $\lambda$.
\item En déduire qu'il existe $G\in\mathrm{GL}_n(\C)$ telle que ${C_P}^\top=GDG^{-1}$, où $D\in\M_n(\C)$ est diagonale et $G^\top$ est une matrice de Vandermonde à expliciter.
\end{enumerate}
\item On suppose que $P$ admet au moins une racine double. Montrer que $C_P$ n'est pas diagonalisable.
\end{enumerate}

\fin
\vspace*{\fill}
\begin{tiny}
Ce sujet est inspiré d'un cours sur la décomposition de Frobenius, disponible à l'adresse suivante: http://www.math.univ-toulouse.fr/~lassere/pdf/vfcomp.pdf
\end{tiny}
\end{document}