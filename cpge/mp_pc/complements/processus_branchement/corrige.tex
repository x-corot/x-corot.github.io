\documentclass[a4paper,11pt]{article}

%-----------------------------------------------------------------

%Packages utilisés
\usepackage{amsmath, amssymb, amsthm}	%Packages mathématiques
\usepackage[utf8]{inputenc}				%Encodage UTF8
\usepackage[francais]{babel}				%Typographie française
\usepackage{fourier}						%Police Fourier
\usepackage{fancyhdr}					%En-tête personnalisé
\usepackage[margin=.8in]{geometry}		%Modification des marges
\usepackage[usenames, dvipsnames]{color, xcolor}	%Ajout de couleurs
\usepackage{titlesec}					%Formattage des titres

%Personnalisation de l'en-tête/pied de page
\pagestyle{fancy}
\renewcommand{\headrulewidth}{0pt}
	%En-tête
	\fancyhead[L]{%
		\strut\rlap{
			\color{vert1}
			\rule[-\dp\strutbox]{\headwidth} {\headheight}
		}%
	  	\sffamily{\color{gris}{ Lycée J-B Corot}}
	}
	\fancyhead[C]{\sffamily{\color{gris}{MP}}}
	
	%Insérer le titre ICI
	\fancyhead[R]{\sffamily{\color{gris}{Processus de branchement}}}
	%--------------------------------------------------
	
	%Pied de page
	\fancyfoot[L]{%
		\strut\rlap{
			\color{vert1}
			\rule[-\dp\strutbox]{\headwidth}{\headheight}
		}%
	  	\sffamily{\color{gris}{}}
	}
	\fancyfoot[C]{%
		\strut\rlap{
			\rule[-\dp\strutbox]{\headwidth}{0pt}
		}%
	  	\sffamily{\color{gris}{\thepage}}
	}
	\fancyfoot[R]{\sffamily{\color{gris}{}}}
	
%Nouvelles commandes
	%Couleurs
	\definecolor{gris}{HTML}{EEEEEE}
	\definecolor{vert1}{HTML}{6AB7C3}
	
	%Numérotation
	\renewcommand{\labelenumi}{\sffamily{\textbf{\color{vert1}\theenumi.}}}
	\renewcommand{\labelenumii}{\sffamily{\textbf{\color{vert1}\theenumii.}}}
	\renewcommand{\labelenumiii}{\sffamily{\textbf{\color{vert1}\theenumiii.}}}
	
	%Mathématiques
	\newcommand{\N}{\mathbf{N}}
	\newcommand{\Z}{\mathbf{Z}}
	\newcommand{\R}{\mathbf{R}}
	\newcommand{\C}{\mathbf{C}}
	\newcommand{\Q}{\mathbf{Q}}
	\newcommand{\K}{\mathbf{K}}
	\newcommand{\U}{\mathbf{U}}
	\renewcommand{\P}{\mathbf{P}}
	\newcommand{\D}{\mathcal{D}}
	\newcommand{\E}{\mathbf{E}}
	\newcommand{\M}{\mathcal{M}}
	\renewcommand{\L}{\mathcal{L}}
	\newcommand{\suite}[2]{{\left({#1}_{#2}\right)}_{#2}}
	
	%Titre
	\newcommand{\titre}[1]{
		\colorbox{vert1}{
		\parbox{\headwidth}{
			\center{\Large{\sffamily\bfseries{\color{gris}{#1}}}}\\
			\vspace*{1em}
		}}
	}
	
	%Corrigé
	\newcommand{\corrige}[1]{\center{
		\colorbox{vert1}{
		\parbox{40em}{
			{\sffamily\bfseries{\color{gris}{Solution}}}\par
			{#1}\\
		}
		}
		}
	}
	
	%Fin
	\newcommand{\fin}{
	\begin{center}
	\vspace*{2em}
	{\color{vert1}\rule{5cm}{0.2pt}}\\
	\vspace*{1em}
			\sffamily\bfseries{\color{vert1}{FIN DU SUJET}}
	\end{center}
	}
	
	%Formattage des titres
	\renewcommand{\thesection}{\Roman{section}}
	\renewcommand{\thesubsection}{\Alph{subsection}}
	\titleformat{\section}
  		{\normalfont\sffamily\Large\bfseries\color{vert1}}
  		{\thesection .}{1em}{}
  	\titleformat{\subsection}
  		{\normalfont\sffamily\large\bfseries\color{vert1}}
  		{\hspace{2em}\thesubsection .}{.5em}{}

%Autres
\setlength\parindent{0pt}	%Pas d'indentation des paragraphes 

%-----------------------------------------------------------------

\begin{document}
\thispagestyle{empty}				%Pas d'en-tête sur la page 1
\pagecolor{gris}						%Couleur de fond
\titre{Processus de branchement}		%Titre du document
\vspace*{1em}

\section*{Objectif}
Les processus de branchement sont des modèles introduits pour étudier le
développement d’une population, dans laquelle les individus se reproduisent
indépendamment les uns des autres.
Ces modèles sont particulièrement utilisés en biologie (étude de la croissance
d’une colonie de bactéries...) et en physique nucléaire, mais trouvent leur origine
dans l’étude, au 19ème siècle, des probabilités d’extinction des noms de familles
illustres en Grande Bretagne (Francis Galton et Henry Watson, 1874). Leur
problème était le suivant :
Si un homme a une probabilité $p_0$ de n’avoir aucun fils, $p_1$ d’avoir un fils, $p_2$
d’en avoir deux, etc ; si chacun de ses fils éventuels est dans le même cas, et ainsi
de suite ; quelle est la probabilité pour qu’à terme cette branche de la famille
s’éteigne ? Plus généralement, comment connaître la probabilité pour qu’il y ait
exactement $k$ individus à la génération $n$ ?

\section*{Notations}
Dans tout le problème, on se place dans un espace probabilisé $(\Omega,\mathcal{A},\P)$.\\

On considère une suite $\suite{Z}{n}$ de variables aléatoires, $Z_n$ modélisant le nombre d'individus de la population à la génération $n$. On fait alors les hypothèses suivantes:
\begin{itemize}
\item[•] Tous les individus, quel que soit la génération à laquelle ils appartiennent,
se comportent a priori de la même façon (i.e. leur nombre de fils a la
même loi de probabilité).
\item[•] Le comportement d’un individu donné n’est en rien influencé par celui
d’autres individus, que ce soit de sa propre génération ou des générations
antérieures.
\item[•] Ce comportement n’est pas non plus affecté par le nombre d’individus.
\end{itemize}
En particulier, les descendances de deux individus distincts sont des phénomènes
indépendants.

On considère une suite $\suite{p}{k}$ de réels positifs ou nuls telle que $\displaystyle\sum_{k=0}^{+\infty}=1$. Chaque individu, au cours du processus, aura une probabilité $p_k$
d’avoir $k$ enfants.\\

On suppose dans la suite du problème que $Z_0=1$, ce qui donne, pour tout $k\in\N$, $\P(Z_1=k)=p_k$.\\

Si $X$ est une variable aléatoire entière et positive sur $(\Omega,\mathcal{A},\P)$, on notera $F_X$, série entière de rayon de
convergence au moins $1$, la fonction génératrice de $X$. On rappelle que la fonction génératrice de $X$ est la
somme de la série entière
\[\forall t\in[-1,1],\ F_X(t)=\displaystyle\sum_{k=0}^{+\infty}\P(X=k)t^k.\]

Si $f:I\rightarrow\R$ est une fonction à valeurs réelles, on dit que $f$ est convexe sur $I$ si
\[\forall x,y\in I,\ \forall\lambda\in[0,1],\ f(\lambda x+(1-\lambda)y)\leq\lambda f(x)+(1-\lambda)f(y).\]

Si $f:I\rightarrow\R$ est une fonction à valeurs réelles, on dit que $f$ est strictement convexe sur $I$ si
\[\forall x,y\in I,\ \forall\lambda\in[0,1],\ f(\lambda x+(1-\lambda)y)<\lambda f(x)+(1-\lambda)f(y).\]

\begin{enumerate}
\setcounter{enumi}{-1}
\item Justifier que, pour tout $n\in\N$, $\omega\in\Omega$,
\[Z_{n+1}(\omega)=\begin{cases}
	0&\text{ si }Z_1(\omega)=0\\
	\displaystyle\sum_{k=1}^{Z_1(\omega)}Z_{n,k}(\omega)&\text{ si }Z_1(\omega)\neq 0
\end{cases}.\]
où les $Z_{n,k}$ sont des variables aléatoires indépendantes de même loi que $Z_n$.

\corrige{La première égalité traduit l'extinction de la population à la $n$-ème génération. D'autre part, si l'on connaît la valeur $i>1$ de $Z_1$, alors les hypothèses en vigueur montrent qu'il suffit de répéter l'expérience de départ $i$ fois (à partir de chaque individu de la première génération) sur $n$ générations pour obtenir $Z_{n+1}(\omega)$, pour tout $\omega\in\Omega$. D'où la formule donnée.}
\end{enumerate}
Ainsi, si l'on connaît $Z_n$, le comportement de la variable $Z_{n+1}$ est absolument indépendant de $Z_0,Z_1,\cdots,Z_{n-1}$.\\

\section{Quelques résultats de convexité}
Dans cette partie, $I$ désigne un intervalle non trivial de $\R$. On se donne une fonction $f:I\rightarrow\R$ de classe $\mathcal{C}^1$.

\begin{enumerate}
\item On suppose que $f$ est de classe $\mathcal{C}^2$ sur $I$. Montrer que $f$ est convexe si, et seulement si, $f''$ est positive.

\corrige{a}
\item On revient au cas général. Montrer que $f$ est convexe si, et seulement si, $f'$ est croissante.
\item On suppose que $f$ est convexe. Montrer que $f$ est au-dessus de sa tangente en tout point de $I$.
\end{enumerate}

\section{Généralités sur les fonctions génératrices}
Dans toute cette partie, on se donne une variable aléatoire $X$ sur $(\Omega,\mathcal{A},\P)$.
\begin{enumerate}
\setcounter{enumi}{3}
\item Montrer que, pour tout $t\in[-1,1]$, $F_X(t)=\E(t^X)$.
\item Montrer que $F_X$ est strictement croissante et convexe sur $[0,1]$.
\item \begin{enumerate}
\item Donner la valeur de $F_X(0)$.
\item Montrer que $F_X$ est continue en $1$.
\item En déduire la valeur de $F_X(1)$.
\end{enumerate}
\item On suppose que $X$ admet une espérance finie. 
\begin{enumerate}
\item Montrer que $\E(X)=\lim\limits_{t\to 1}{F_X}'(t)$.
\item En déduire que $\E(X)={F_X}'(1)$.
\end{enumerate}
\item On suppose que $X$ admet une variance finie. En particulier, $X$ admet une espérance finie, donc les résultats précédents restent vrais.
\begin{enumerate}
\item Déterminer $\lim\limits_{t\to 1}{F_X}''(t)$.
\item Montrer alors que $\mathrm{Var}(X)={F_X}''(1)-\left({F_X}'(1)\right)^2+{F_X}'(1)$.
\end{enumerate}
\item Soit $Y$ une variable aléatoire indépendante de $X$. 
\begin{enumerate}
\item Montrer que $F_{X+Y}=F_XF_Y$.
\item Montrer que, pour tout $n\in\N$, $X_1,\cdots,X_n$ des variables aléatoires mutuellement indépendantes, de somme $S_n$, on a $F_{S_n}=F_{X_1}\cdots F_{X_n}$.
\end{enumerate}
\end{enumerate}

\section{Cas des processus de branchement}
\subsection{Fonctions génératrices des variables $Z_n$}
On considère la fonction $F=F_{Z_1}$, donnée par l'expression $F(t)=\displaystyle\sum_{k=1}^{+\infty}p_kt^k$, pour tout $t\in[-1,1]$. On définit alors la suite de fonctions $\suite{F}{n}$ par la relation de récurrence \[F_0=\mathrm{id}_{[-1,1]}\text{ et }F_{n+1}=F\circ F_n\text{ pour tout }n\in\N.\]
On se propose de montrer que, pour tout $n\in\N$, $F_{Z_n}=F_n$. Pour cela, on procède par récurrence sur $n$.
\begin{enumerate}
\setcounter{enumi}{9}
\item Montrer que $F_{Z_0}=F_0$ et que $F_{Z_1}=F_1$.
\item Soit $n\geq 2$. Supposons le résultat vrai pour tout $m<n$. Pour tout $i\in\N$, on définit la variable aléatoire $Y_{n,i}$ par
\[\forall k\in\N,\ \P(Y_{n,i}=k)=\P(Z_n=k\vert Z_1=i).\]
\begin{enumerate}
\item Montrer que, pour tout $k\in\N$, $\P(Z_n=k)=\displaystyle\sum_{i\in I}\P\left(Y_{n,i}=k\right)p_i$, avec $I=\{i\in\N\ \vert\ p_i>0\}$.
\item Montrer que, pour tout $i\in\N$, $t\in[-1,1]$, $F_{Y_{n,i}}(t)=\left( F_{n-1}(t)\right)^i$.
\item Montrer que, pour tout $t\in[-1,1]$, $F_{Z_n}(t)=\displaystyle\sum_{i\in I}p_i\left( F_{n-1}(t)\right)^i$.
\item Conclure.
\end{enumerate}
\end{enumerate}

\subsection{Un exemple}
On suppose que, pour tout $i\in\N$, $p_i=\frac{1}{2^{i+1}}$.
\begin{enumerate}
\setcounter{enumi}{11}
\item Vérifier que l'on définit bien une loi de probabilité.
\item Déterminer $F_1$.
\item \begin{enumerate}
\item Vérifier que, pour tout $t\in[0,1[$, $F_n(t)\neq 1$. On peut alors poser $a_n=\frac{1}{F_n-1}$ pour tout $n\in\N$.
\item Montrer que, pour tout $t\in[0,1[$, la suite ${\left( a_n(t)\right)}_n$ est arithmétique.
\item En déduire que, pour $t\in[0,1[$, $F_n(t)=\frac{n-(n-1)t}{n+1-nt}$.
\end{enumerate}
\item Exprimer, pour $(n,k)\in\N^2$, $\P(Z_n=k)$.
\end{enumerate}

\subsection{Probabilité d'extinction}
Dans la suite du problème, on supposera que $p_0>0$ et que $Z_1$ possède une espérance finie.
\begin{enumerate}
\setcounter{enumi}{15}
\item Montrer que la probabilité d'extinction est égale à la limite de la suite ${\left( F_n(0)\right)}_n$. On notera $\alpha$ cette limite.
\item On note $x_0$ le plus petit point fixe de $F$.
\begin{enumerate}
\item Justifier l'existence de $x_0$.
\item Montrer que $\alpha\leq x_0$.
\item En déduire que $\alpha=x_0$.
\end{enumerate}
\item On suppose que $\E(Z_1)< 1$. Montrer que $1$ est l'unique point fixe de $F$ et en déduire la valeur de $\alpha$.
\item On suppose que $\E(Z_1)= 1$.
\begin{enumerate}
\item Montrer qu'il existe $n\geq 2$ tel que $p_n>0$.
\item En déduire que $F$ est strictement convexe.
\item Déterminer la valeur de $\alpha$.
\end{enumerate}
\item On suppose que $\E(Z_1)> 1$. On considère la fonction $\varphi:\ t\in[0,1[\mapsto\frac{1-F(t)}{1-t}$.
\begin{enumerate}
\item Montrer que $\varphi$ est strictement croissante.
\item En déduire l'existence d'un unique point fixe $x_0$ de $F$ sur $[0,1[$. On a donc $\alpha=x_0<1$.
\end{enumerate}
\end{enumerate}

\section{Cas sur-critique ($\E(Z_1)>1$)}
On suppose ici que $\E(Z_1)>1$. Dès lors, $\alpha<1$.\\

Dans toute cette partie, on fixe un entier naturel non nul $k$ et on définit de manière analogue aux $Z_n$ une suite de variables aléatoires $\suite{Z'}{n}$, vérifiant cependant ${Z'}_0=k$. On admet que, hormis cette dernière hypothèse, les résultats démontrés précédemment pour les $Z_n$ restent valables pour les ${Z'}_n$.\\

On définit de plus une suite $\suite{u}{n}$ par $u_0=1$ et, pour tout $n\in\N^*$,
\[u_{n}=\P\left({Z'}_n=k\text{ et }{Z'_i}\neq k\text{ pour tout }0<i<n\right),\]
et $u=\displaystyle\sum_{n=1}^{+\infty}u_n$.\\

Pour tout $n,r\in\N^*$, on définit $u_n^{(r)}$ comme la probabilité pour que la suite $\suite{Z'}{n}$ prenne la valeur $k$ pour la $r$-ième fois au rang $n$. \\

Enfin, on pose $U(t)=\displaystyle\sum_{n=1}^{+\infty}u_nt^n$ et $U_r(t)=\displaystyle\sum_{n=1}^{+\infty}u_n^{(r)}t^n$, pour tout $t\in[-1,1]$ et $r\in\N^*$.

\begin{enumerate}
\setcounter{enumi}{20}
\item Montrer que, pour tout $n\in\N$, $\E(Z_n)=\alpha^n$.
\item Vérifier que les séries $\displaystyle\sum_{n\geq 1}u_nt^n$ et $\displaystyle\sum_{n\geq 1}u_n^{(r)}t^n$ convergent quand $t\in[-1,1]$.
\item 
\begin{enumerate}
\item Justifier que $u$ est bien défini.
\item Montrer que la probabilité que la suite ${\left({Z'}_n\right)}_{n\in\N^*}$ ne prenne pas la valeur $k$ est non nulle. Ainsi, on a $u<1$.
\item Ce résultat subsiste-t-il si l'on suppose que $p_0=0$ ?
\end{enumerate}
\item \begin{enumerate}
\item Soit $n\in\N^*$ et $r$ un entier supérieur ou égal à $2$. Montrer la relation
\[u_n^{(r)}=\displaystyle\sum_{i=1}^{n-1}u_iu_{n-i}^{(r-1)}.\]
\item En déduire que, pour tout entier $r$ strictement positif, $U_r=U^r$ ($U^r$ désigne $U\times U\times\cdots\times U$ $r$ fois).
\end{enumerate}
\item
\begin{enumerate}
\item Montrer que la probabilité que la suite $\suite{Z'}{n}$ prenne la valeur $k$ une infinité de fois est nulle.
\item Montrer que la probabilité que la suite $\suite{Z}{n}$ prenne la valeur $k$ une infinité de fois est nulle.
\end{enumerate}
\item Soit $\suite{A}{n}$ une suite d'événements, tous de probabilité $1$. Que peut-on dire de $\P\left(\displaystyle\bigcap _{n\in\N}A\right)$ ?
\item Montrer que $\P\left(\lim\limits_{n\to+\infty}Z_n=+\infty\right)=1-\alpha$.
\end{enumerate}

On a ainsi montré que, contrairement au cas sous-critique ($\E(Z_1)\leq 1$) où la population est vouée à s'éteindre presque sûrement, la suite $\suite{Z}{n}$ converge vers $0$ avec une probabilité $\alpha<1$. Le cas contraire, elle diverge vers $+\infty$ avec une probabilité $1-\alpha$.

\fin
\vspace*{\fill}
\begin{tiny}
Ce sujet est largement inspiré de l'excellent document disponible à l'adresse suivante: http://culturemath.ens.fr/maths/pdf/proba/watson.pdf (que j'ai eu l'occasion d'étudier au cours de l'épreuve d'ADS de l'École polytechnique), ainsi que la remarquable épreuve de Mathématiques 1 – PSI – Centrale 2015.
\end{tiny}
\end{document}