\documentclass[a4paper,11pt]{article}

%-----------------------------------------------------------------

%Packages utilisés
\usepackage{amsmath, amssymb, amsthm}	%Packages mathématiques
\usepackage[utf8]{inputenc}				%Encodage UTF8
\usepackage[francais]{babel}				%Typographie française
\usepackage{fourier}						%Police Fourier
\usepackage{fancyhdr}					%En-tête personnalisé
\usepackage[margin=.8in]{geometry}		%Modification des marges
\usepackage[usenames, dvipsnames]{color, xcolor}	%Ajout de couleurs
\usepackage{titlesec}					%Formattage des titres

%Personnalisation de l'en-tête/pied de page
\pagestyle{fancy}
\renewcommand{\headrulewidth}{0pt}
	%En-tête
	\fancyhead[L]{%
		\strut\rlap{
			\color{vert1}
			\rule[-\dp\strutbox]{\headwidth} {\headheight}
		}%
	  	\sffamily{\color{gris}{ Lycée J-B Corot}}
	}
	\fancyhead[C]{\sffamily{\color{gris}{MP}}}
	\fancyhead[R]{\sffamily{\color{gris}{Autour des endomorphismes nilpotents}}}
	
	%Pied de page
	\fancyfoot[L]{%
		\strut\rlap{
			\color{vert1}
			\rule[-\dp\strutbox]{\headwidth}{\headheight}
		}%
	  	\sffamily{\color{gris}{}}
	}
	\fancyfoot[C]{%
		\strut\rlap{
			\rule[-\dp\strutbox]{\headwidth}{0pt}
		}%
	  	\sffamily{\color{gris}{\thepage}}
	}
	\fancyfoot[R]{\sffamily{\color{gris}{}}}
	
%Nouvelles commandes
	%Couleurs
	\definecolor{gris}{HTML}{EEEEEE}
	\definecolor{vert1}{HTML}{6AB7C3}
	
	%Numérotation
	\renewcommand{\labelenumi}{\sffamily{\textbf{\color{vert1}\theenumi.}}}
	\renewcommand{\labelenumii}{\sffamily{\textbf{\color{vert1}\theenumii.}}}
	
	%Mathématiques
	\newcommand{\N}{\mathbf{N}}
	\newcommand{\Z}{\mathbf{Z}}
	\newcommand{\R}{\mathbf{R}}
	\newcommand{\C}{\mathbf{C}}
	\newcommand{\Q}{\mathbf{Q}}
	\newcommand{\K}{\mathbf{K}}
	\newcommand{\U}{\mathbf{U}}
	\newcommand{\D}{\mathcal{D}}
	\newcommand{\E}{\mathcal{E}}
	\newcommand{\M}{\mathcal{M}}
	\renewcommand{\L}{\mathcal{L}}
	\newcommand{\suite}[2]{{\left({#1}_{#2}\right)}_{#2}}
	
	%Titre
	\newcommand{\titre}[1]{
		\colorbox{vert1}{
		\parbox{\headwidth}{
			\center{\Large{\sffamily\bfseries{\color{gris}{#1}}}}\\
			\vspace*{1em}
		}}
	}
	
	%Corrigé
	\newcommand{\corrige}[1]{
		\colorbox{vert1}{
		\parbox{\headwidth}{
			{\sffamily\bfseries{\color{gris}{Solution}}}\par
			{#1}\\
		}
		}
		
	}
	
	%Formattage des titres
	\renewcommand{\thesection}{\Roman{section}}
	\renewcommand{\thesubsection}{\Alph{subsection}}
	\titleformat{\section}
  		{\normalfont\sffamily\Large\bfseries\color{vert1}}
  		{\thesection .}{1em}{}
  	\titleformat{\subsection}
  		{\normalfont\sffamily\large\bfseries\color{vert1}}
  		{\hspace{2em}\thesubsection .}{.5em}{}

%Autres
\setlength\parindent{0pt}	%Pas d'indentation des paragraphes 

%-----------------------------------------------------------------

\begin{document}
\thispagestyle{empty}	%Pas d'en-tête sur la page 1
\pagecolor{gris}			%Couleur de fond
\titre{Autour des endomorphismes nilpotents}
\vspace*{1em}
\section*{Notations}
Soit $n$ un entier naturel non nul et $(E,\langle\vert\rangle)$ un espace préhilbertien de dimension $n$. On note $\L(E)$ l'ensemble des applications linéaires de $E$ dans lui-même.

On note $\M_n(\R)$ l'espace vectoriel des matrices carrées d'ordre $n$ à coefficients complexes, $I_n$ la matrice identité de $\M_n(\R)$ et $\mathrm{GL}_n(\R)$ l'ensemble des matrices inversibles de $\M_n(\R)$.

Une application $u\in\L(E)$ est dite nilpotente d'indice $p$ si $p$ est le plus petit entier strictement positif pour lequel $N^p=0$. Pour $u\in\L(E)$, on note $\chi_u$ le polynôme caractéristique de $M$ et $\mathrm{Sp}(u)$ l'ensemble de ses valeurs propres.

On pose \[J_n=\begin{pmatrix}
0&1&&\\
&\ddots&\ddots&\\
&&\ddots&1\\
&&&0

\end{pmatrix}\in\M_n(\R).\]

Dans tout le problème, on considèrera une application $u\in\L(E)$ nilpotente d'indice $p$.

\section{Réduction de Jordan des endomorphismes nilpotents}
	\subsection{Une majoration de $p$}
	\begin{enumerate}
	\item
	\begin{enumerate}
		\item Donner un polynôme annulateur de $u$.
		\item Déterminer les valeurs propres de $u$.
	\end{enumerate}
	\item Soit $v\in\L(E)$. On suppose que $\mathrm{Sp}(v)=\{0\}$. Montrer que $v$ est nilpotente.
	\end{enumerate}
	On a ainsi montré qu'une application $v\in\L(E)$ est nilpotente si, et seulement si, $\chi_v$ est scindé et $\mathrm{Sp}(v)=\{0\}$.
	\begin{enumerate}
	\setcounter{enumi}{2}
	\item En déduire que $p\leq n$.
	\end{enumerate}
	\subsection{Le cas $p=n$}
	\begin{enumerate}
	\setcounter{enumi}{3}
	\item \begin{enumerate}
	\item Justifier l'existence de $x\in E$ tel que $u^{n-1}(x)\neq 0$.
	\item Montrer alors que la famille $\mathcal{B}=(u^{n-1}(x),\cdots,u(x),x)$ est une base de $E$.
	\end{enumerate}
	\item En déduire la matrice de $u$ dans $\mathcal{B}$.
	\end{enumerate}
	\subsection{Le cas $p<n$}
	\begin{enumerate}
	\setcounter{enumi}{5}
	\item Montrer qu'il existe $x\in E$ tel que la famille $(u^{p-1}(x),\cdots,u(x),x)$ soit libre dans $E$.
	\item En déduire que $F=\mathrm{Vect}\left(u^{p-1}(x),\cdots,u(x),x\right)$ est stable par $u$.
	\item \begin{enumerate}
	\item Montrer que $F^\perp$ est stable par $u$.
	\item Montrer que $E=F\oplus F^\perp$.
	\item Montrer que les restrictions $u_{\vert F}$ et $u_{\vert F^\perp}$ de $u$ à $F$ et $F^\perp$ sont nilpotentes.
	\end{enumerate}
	\item Montrer qu'il existe une base $\mathcal{B}$ de $E$ dans laquelle la matrice de $u$ est de la forme $\begin{pmatrix}
	J_p&0\\0&M
	\end{pmatrix}$.
	\end{enumerate}
	
	\subsection{Caractéristiques de la décomposition de Jordan}
	\begin{enumerate}
	\setcounter{enumi}{9}
	\item Montrer qu'il existe $p_1\geq\cdots\geq p_s\in\N^*$ ainsi qu'une base $\mathcal{B}$ de $E$ dans laquelle la matrice de $u$ est de la forme $\begin{pmatrix}
	J_{p_1}&\\&\ddots&\\&&J_{p_s}
	\end{pmatrix}$. Cette forme est appelée la décomposition de Jordan de $u$.
	\item En déduire la valeur de $p$.
	\item Justifier l'existence d'une famille libre $(x_1,\cdots,x_s)\in E^s$ tel que pour $1\leq i\leq s$, $u^{p_i-1}(x_i)\neq 0$.
	\item Montrer que $(x_1,\cdots,x_s)$ est une base de $\mathrm{Ker}(u)$. En déduire la valeur de $s$.
	\end{enumerate}
	Pour $0\leq k\leq s$, on note $d_k=\mathrm{dim}\left(\mathrm{Ker}u^k\right)$.
	\begin{enumerate}
	\setcounter{enumi}{13}
	\item Montrer que $(x_1,\cdots,x_s,u(x_1),\cdots,u(x_s))$ est une base de $\mathrm{Ker}(u^2)$. En déduire $\mathrm{Card}\left(\{p_i\vert 1\leq i\leq s\text{ et }p_i\geq 2\}\right)$.
	\item Montrer que pour $1\leq k\leq s$, $\mathrm{Card}\left(\{p_i\vert 1\leq i\leq s\text{ et }p_i\geq k\}\right)=d_{k}-d_{k-1}$.
	\end{enumerate}
	
	\subsection{Généralisation aux matrices quelconques}
	On considère $v\in\L(E)$ un endomorphisme de $E$ tel que $\chi_v$ soit scindé. On note alors $\chi_v=\displaystyle\prod_{k=1}^r(X-\lambda_k)^{\alpha_k}$, les $\lambda_k$ étant des réels deux à deux distincts et les $\alpha_k$ étant des entiers naturels non nuls.
	\begin{enumerate}
	\setcounter{enumi}{15}
	\item Montrer que $E=\displaystyle\bigoplus_{k=1}^rC_k$, où $C_k=\mathrm{Ker}\left(v-\lambda_k\mathrm{id}_E\right)^{\alpha_k}$. On appelle sous-espaces caractéristiques de $u$ les $C_k$.
	\item Montrer que, pour tout $1\leq k\leq r$, $C_k$ est stable par $v$.
	\item Montrer que, pour tout $1\leq k\leq r$, $v_k=v_{\vert C_k}$ peut s'écrire $v_k=\lambda_k\mathrm{id}_E+n_k$, où $n_k$ est un endomorphisme nilpotent de $C_k$.
	\item Montrer qu'il existe $p_1\geq\cdots\geq p_s\in\N^*$, ainsi qu'une base $\mathcal{B}$ de $E$ dans laquelle la matrice de $v$ est de la forme \[\begin{pmatrix}
	\lambda_1I_{p_1}+J_{p_1}&\\&\ddots&\\&&\lambda_sI_{p_s}+J_{p_s}
	\end{pmatrix}.\]
	\item Justifier que l'on peut déterminer la valeur des $p_i$, uniquement en fonction de $u$. En déduire l'unicité des $p_i$.
	\item Montrer alors que deux matrices $A$ et $B$ sont semblables dans $\C$ si, et seulement si 
	\[\forall (\lambda,k)\in\C\times\N^*,\ \mathrm{rg}(A-\lambda I_n)^k=\mathrm{rg}(B-\lambda I_n)^k.\]
	\end{enumerate}
	
	\subsection{Commutant d'un endomorphisme}
\end{document}